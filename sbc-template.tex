\documentclass[12pt]{article}

% ---
% PACOTES
% ---

% ---
% Pacotes fundamentais 
% ---
\usepackage{sbc-template}
\usepackage{graphicx,url}
\usepackage[utf8]{inputenc}
\usepackage[brazil]{babel}
% ---


\usepackage{caption}
\usepackage{listings}
\usepackage{xcolor}

\definecolor{codegreen}{rgb}{0,0.6,0}
\definecolor{codegray}{rgb}{0.5,0.5,0.5}
\definecolor{codepurple}{rgb}{0.58,0,0.82}
\definecolor{backcolour}{rgb}{0.95,0.95,0.92}

\lstdefinestyle{mystyle}{
    backgroundcolor=\color{backcolour},   
    commentstyle=\color{codegreen},
    keywordstyle=\color{magenta},
    numberstyle=\tiny\color{codegray},
    stringstyle=\color{codepurple},
    basicstyle=\ttfamily\footnotesize,
    breakatwhitespace=false,         
    breaklines=true,                 
    captionpos=t,                    
    keepspaces=true,                 
    numbers=left,                    
    numbersep=5pt,                  
    showspaces=false,                
    showstringspaces=false,
    showtabs=false,                  
    tabsize=2
}

\lstset { %s
    style=mystyle,
    %language=C++,
    extendedchars=true,
    %inputencoding=brazil
  	%basicstyle=\footnotesize,% basic font setting
  	%basicstyle=\ttfamily 	
    literate=
    {á}{{\'a}}1
    {à}{{\`a}}1
    {ã}{{\~a}}1
    {â}{{\~a}}1
    {é}{{\'e}}1
    {ê}{{\^e}}1
    {í}{{\'i}}1
    {ó}{{\'o}}1
    {õ}{{\~o}}1
    {ú}{{\'u}}1
    {ç}{{\c{c}}}1
    {Á}{{\'A}}1
    {À}{{\`A}}1
    {Ã}{{\~A}}1
    {É}{{\'E}}1
    {Ê}{{\^E}}1
    {Í}{{\'I}}1
    {Ó}{{\'O}}1
    {Õ}{{\~O}}1
    {Ú}{{\'U}}1
    {Ç}{{\c{C}}}1
  }



\renewcommand{\lstlistingname}{Registro de Saída}
% --- 
% SEUS AJUSTES DEVEM SER FEITOS ABAIXO DESSE COMENTÁRIO
% --- 


% ---
% Pacotes de citações
% ---
%\usepackage[brazilian,hyperpageref]{backref}	 % Paginas com as citações na bibl
%\usepackage[alf]{abntex2cite}	% Citações padrão ABNT


\sloppy
\title{Proto3d: Um estudo de como tornar uma simulação computacional acessível}

\author{Róger Matheus Lasch \inst{1}, José Antônio de Figueiredo\inst{1}}

\address{Curso Bacharel em Ciência da Computação -- Instituto Federal Sul-rio-grandense (IFSUL)\\
	Passo Fundo -- RS -- Brasil
	\email{rogerlasch@gmail.com, josefigueiredo@ifsul.edu.br}
}

\begin{document} 

\maketitle

\begin{abstract}
\end{abstract}
     
\begin{resumo} 
  Aqui vai o resumo
\end{resumo}

\section{Introdução}
O objetivo primordial deste trabalho é desenvolver uma aplicação dedicada ao estudo da simulação de corpos rígidos, que servem para modelar o movimento e a iteração entre diversos objetos. Durante o processo de desenvolvimento dessa aplicação, surgiu a necessidade de considerar como testar os componentes construídos sem depender exclusivamente de recursos visuais.
 Embora a visualização da colisão entre corpos rígidos seja  comumente realizada de maneira visual, este estudo busca criar uma abordagem alternativa, especialmente projetada para atender às necessidades de pessoas com deficiência visual. O método proposto prioriza a rastreabilidade das propriedades críticas do problema em detrimento da representação visual, o que torna os resultados acessíveis e compreensíveis tanto para o público-alvo quanto para outros leitores.
Neste trabalho são apresentados os principais resultados da construção da aplicação Proto3d, um simulador de movimento e colisão de corpos rígidos, que mostra os resultados em uma saída acessível aos leitores de tela disponíveis no mercado. Além disso o Proto3d é uma ferramenta configurável, que permite adicionar e personalizar o mundo e os objetos para a simulação.

O artigo está organizado nas seguintes seções:
2. Simulação Computacional
Nesta seção, abordaremos conceitos fundamentais relacionados à simulação computacional, bem como aspectos pertinentes às formas geométricas e detecção de colisão.
3. Materiais e Métodos
A seção de Materiais e Métodos oferece uma descrição detalhada dos principais componentes do Proto3d, incluindo corpos rígidos, ambiente de simulação, fase ampla e outros elementos cruciais para a compreensão do sistema em estudo.
4. Resultados Obtidos
Nesta seção, apresentaremos os resultados obtidos até o momento com o Proto3d, além de fornecer um exemplo prático de um cenário de teste, demonstrando como o sistema se comportou e o que foi observado durante as simulações.
5. Considerações Finais
Na seção de Considerações Finais, concluiremos o artigo, resumindo os principais achados e destacando possíveis direções futuras para a pesquisa.


o que ter na introdução....

\section{Simulação Computacional}

Segundo \cite{duran2018computer} a simulação computacional é uma técnica amplamente utilizada em diversas áreas, como por exemplo no estudo de sistemas complexos que por diversos fatores, não podem ser facilmente reproduzidos.

A simulação permite avaliar diversos cenários de uma pesquisa, apenas alterando suas variáveis de controle, reduzindo assim os custos e os riscos de um experimento físico. Conforme \cite{marschner2018}, essa técnica pode ser aplicada em diversas áreas como biologia, engenharia, física, química, entre outras.

Para construir uma simulação, é necessário uma representação matemática e/ou computacional do comportamento do sistema a ser simulado.

\subsection{Formas Geométricas e Movimento}
A técnica usada para representar uma forma geométrica computacionalmente dependerá do nível de detalhamento desejado \cite{pharr2016}. É possível fazer esta representação de diversas maneiras, por exemplo: forma vetorial, paramétrica ou de forma poligonal. Neste trabalho, adota-se a representação vetorial, com coordenadas cartesianas tridimensionais.

Para fins de simplificação, este trabalho limita-se a modelar objetos na forma de esferas e caixas. Esferas são modeladas com um ponto (centro da esfera) e um raio, enquanto caixas são modeladas com um ponto mínimo e um ponto máximo, que representam o canto inferior esquerdo, e o canto superior direito da caixa.

No contexto desta simulação, os objetos podem ser categorizados em duas classes principais: Objetos estáticos e objetos dinâmicos. Os objetos estáticos não se movem por conta própria e nem sob influência de forças externas. Em contrapartida, os objetos dinâmicos podem se mover por conta própria, bem como em resposta a forças externas.

\subsection{Colisão e Detecção de Colisão}

Uma colisão ocorre quando dois ou mais objetos entram em contato, compartilhando algum espaço do ambiente simulado~\cite{ericson2004real}. Segundo~\cite{bourg2013physics}, a colisão envolve cálculos para determinar as consequências/efeitos desse contato.

Segundo~\cite{bourg2013physics} a detecção de colisão é um problema de geometria computacional, que procura determinar \emph{se} e \emph{onde}, dois ou mais objetos colidiram. Pode ser detectada de diversas maneiras, dependendo da natureza dos objetos envolvidos~\cite{foley1996}, por exemplo: 1) se a distância entre os dois objetos está abaixo de uma determinada tolerância; 2) se os objetos estiverem sobrepondo-se em um ou mais pontos.

O processo para detecção de colisão é dividido em fase de previsão e de identificação. A previsão busca objetos que possivelmente estejam colidindo, enquanto a identificação localiza as colisões ocorridas. A separação em duas fases distintas garante um consumo racional do recurso computacional. 

Após detectar uma colisão entre dois objetos, torna-se necessário resolver o evento, atualizando a posição da simulação. Em simulações que não requerem realismo, apenas a correção das posições dos objetos é suficiente. Já em simulações mais avançadas, outros métodos podem ser necessários, como a resolução de colisão por impulso discutida em~\cite{bourg2013physics} e~\cite{baraff1992dynamic}, onde as características físicas dos objetos são considerados para que a reação seja a mais adequada possível.

%########
\section{Materiais e Métodos}
-> objetivo do simulador
-> interface de configuração (falar somente do que pode ser configurado.)
-> principais classes (sem detalhar todos os atributos)
-> recursos utilizados

%########

O Proto3d é um programa que permite realizar simulações de corpo rígido.
Ele possui algumas propriedades configuráveis tais como gravidade, e limites do ambiente.
No ambiente, objetos são adicionados para se realizar a simulação. Estes objetos são normalmente chamados de corpos rígidos.
Os corpos rígidos, são compostos com diversas propriedades como restituição, posição, velocidade, orientação, forma, e outras propriedades dependendo do nível de detalhamento que se deseja.
O simulador, é constituído  por diversos componentes:

\begin{itemize}
\item Classe Mundo: Esta classe representa o ambiente virtual onde a simulação ocorre, sendo o núcleo central de todo o sistema.
\item Classe Corpo Rígido: Esta classe é responsável por representar objetos no ambiente, podendo ser objetos dinâmicos ou estáticos.
\item Classe Pré-filtragem: No início de cada quadro, essa classe realiza uma filtragem em uma lista de objetos, preparando-os para a simulação de movimento.
\item Classe Integração Numérica: Aqui, as equações de movimento são resolvidas para cada objeto previamente preparado pela classe de pré-filtragem.
\item Classe Fase Ampla: Nesta classe, ocorre a verificação de uma lista de pares de objetos cujas caixas delimitadoras (AABBS) estão se sobrepondo, o que pode indicar uma possível colisão entre os objetos.
\item Classe Fase Estreita: Essa classe aplica testes específicos de interseção entre geometrias para confirmar ou descartar colisões entre os objetos previamente identificados pela fase ampla.
\item Classe Resolvedor de Colisões: Aqui, a ação é tomada para separar objetos que estejam em sobreposição, trabalhando com o conjunto de objetos retornados pela fase estreita.
\item Classe Despacho de Eventos: Nesta classe, notificações são enviadas ao usuário para informar sobre os eventos ocorridos no quadro atual da simulação.
\end{itemize}

\section{Resultados Obtidos}
O principal resultado é a aplicação Proto3d, que executa simulações de movimento e colisão de corpos rígidos em duas ou três dimensões, dentro do ambiente configurado pelo usuário. 

A simulação do movimento é implementada na dinâmica newtoniana, enquanto a detecção de colisão é implementada em uma estrutura de dados \emph{Octree}, com algoritmos \emph{Broad Phase} \emph{Narrow Phase} para previsão e identificação, respectivamente.

Além disso, o Proto3d é uma ferramenta acessível às pessoas com deficiência visual, desta forma os resultados das simulações são apresentados no Registro de Saída, um log acessível aos leitores de tela disponíveis no mercado.

\subsection{Simulador Proto3d}
O Proto3d é um simulador de movimento e colisão de corpos rígidos. Para executar uma simulação, a ferramenta permite dimensionar e personalizar um ambiente (mundo), bem como inserir e personalizar corpos rígidos estáticos ou dinâmicos. Os seguintes parâmetros são configuráveis:
\begin{itemize}
	\item \textbf{Mundo}: gravidade, limites físicos do ambiente e passo de tempo;
	\item \textbf{Corpos rígidos estáticos}: forma, posição e orientação;
	\item \textbf{Corpos rígidos dinâmicos}: forma, massa, posição, velocidade e coeficiente de restituição\footnote{O coeficiente de restituição é um valor que varia de de 0 a 1 e especifica o quanto de energia cinética é conservada no objeto após uma colisão};
\end{itemize}

+++ avaliar esta explicação ++++
A localização e o tamanho corpos rígidos é definida pela posição e espaço que ocupa. Posição indica a coordenada do ponto central do objeto, enquanto que o espaço que ocupa será definido pelo intervalo entre duas coordenadas no referido eixo.
+++++ até aqui +++++

A execução da simulação é apresentada em formato texto no Registro de Saída, sendo que esta versão do Proto3d não dispõe de interface gráfica. Para validação das simulações, foram executados alguns testes com grandezas conhecidas. 

Para este estudo, optou-se em restringir o formato dos objetos em esferas e caixas, apresentando os resultados da simulação para um destes testes.

\subsection{Registro de Saída}
O Proto3d apresenta os resultados de cada simulação no Registro de Saída, em formato de log de texto e também armazena em um arquivo para análise posterior. O registro é organizado em duas partes principais: \emph{Visão Geral} e \emph{Rastreio do Movimento}.

\textbf{Visão Geral}: Apresenta as configurações gerais do ambiente simulado, como: gravidade, especificação dos limites do mundo e a lista de objetos simulados. Por convenção, as grandezas na simulação podem ser lidas como: dimensão em metros (m), massa em quilogramas (kg), tempo em segundos (s), velocidade em metros/segundo (m/s) e força de movimento em $metros/segundo^{2}$ ($m/s^{2}$).

\textbf{Rastreio do Movimento}: Apresenta informações das iterações do simulador a cada passo de tempo. Os registros possuem informações conforme a legenda:

\begin{itemize}
	\item T: Tempo decorrido desde o início da simulação até este registro;
	\item Dir: Trajetória do objeto dinâmico conforme a legenda: s (direção sul), n (direção norte), o (direção oeste), l (direção leste), c (direção acima/para cima), b (direção abaixo/para baixo); É possível ocorrer combinações entre as trajetórias básicas. 
	\item Pos: Posição do objeto neste instante (em metros) nos eixos X (horizontal), Y (profundidade) e Z (vertical);
	\item Vel: Velocidade do objeto neste instante (m/s).
\end{itemize}

\subsection{Testando uma simulação}
O funcionamento do Proto3d é demonstrado com a simulação de um objeto dinâmico que, atraído pela gravidade, cai e colide com um objeto estático. Depois da colisão com a caixa, sob efeito de uma nova força horizontal, o objeto dinâmico é movido novamente. As etapas da simulação deste teste estão detalhadas a seguir.  

O Quadro~\ref{lg:init_world} mostra o Registro de Saída de inicialização do ambiente, para o teste proposto. Para fins de adequação ao formato de artigo, algumas linhas do Registro de Saída original poderão ser omitidas nos quadros apresentados.

O registro inicia mostrando Tempo atual=0.00s e Último passo de tempo=0.00s, indicando que a simulação ainda não começou. A seguir mostra o parâmetro Gravidade=True e Gravidade: 0.00:0.00:-9.81, indicando que a gravidade está ativada, com força $9,81 m/s^{2}$ no eixo Z, não existindo forças nos eixos X e Y. O valor é negativo para que a força "atraia" os objetos para baixo. 

/* revisar aqui:  inicio VERIFICAR SISTEMA DE COORDENADAS */ 

Na terceira linha são mostrados os limites do ambiente, tendo um valor inicial e um final para cada eixo. As dimensões do espaço de simulação são 1000000m nos eixos X(largura) e Y (profundidade). O eixo Z (altura) 100000m. 

Nas linhas 5 e 6 estão especificados o chão (piso) do ambiente de simulação, representado por uma caixa. O chão do mundo encontra-se na posição 


Importante destacar que este chão não representa o limite inferior do mundo, ou seja, nesta simulação há espaço vazio depois do chão. 

O chão encontra-se 


/* revisar aqui:   fim */

\begin{lstlisting}[frame=single,caption=Registro de Saída: Inicialização do ambiente\label{lg:init_world}]
	Tempo atual: 0.00s; Último passo de tempo: 0.00; 
	Gravidade: True; 	Gravidade: 0.00:0.00:-9.81
	Limites: 0.00:1000000.00,0.00:1000000.00,-10000.00:1000000.00
	
	Nome: Chão; Pos: 500000.00:500000.00:-5000.00; Forma Geom: Caixa
	AABB -> X: 0.00:1000000.00; Y: 0.00:1000000.00; Z: -10000.00:0.00 
	
	Nome: Caixa estática; Pos: 35.00:35.00:5.00; Forma Geom: Caixa
	AABB -> X: 10.00:60.00; Y: 10.00:60.00; Z: 0.00:10.00; 
	
	Nome: Esfera dinâmica; Forma Geo: esfera; 
	Centro: 10.00:10.00:50.00; Raio: 2.50
	Masss: 5.00; Restituição: 0.25
	AABB -> X: 7.50:12.50; Y: 7.50:12.50; Z: 47.50:52.50
\end{lstlisting}

A seguir estão as especificações do objeto estático denominado caixa. Na linha 8 está a localização do ponto central da caixa, na coordenada X=35,Y=35,Z=5. A linha 9 mostra os valores do AABB, que marcam os limites da caixa. X=10:60 indica início no ponto 10 e fim no ponto 60; Y=10:60 indica início no ponto 10 e fim no ponto 60; Z=0:10 que indica início no ponto 0 e fim no ponto 10;

 nos 3 eixos, sendo que cada eixo é representado por um par de valores, com a posição inicial e a posição final (cantos da caixa). As medidas apresentadas informam o intervalo 10:60 no eixo X. 

que no eixo XEm X está no intervalo 10:60, ou seja é está limitado entre os pontos 10m e 60m, conferindo uma largura de 50m. Em Y está no intervalo 10:60, conferindo uma largura de 50m. Já em Z está no intervalo 0:10, conferindo uma altura de 10m.


Na linha 8 estão as especificações do objeto estático, denominado caixa. Os valores indicam que ponto central da caixa está a 35m da origem no eixo X, 35m da origem no eixo Y e tem altura 5m, a partir da origem no eixo Z. Na linha 9 está especificado o AABB do objeto, indicando os limites do mesmo nos 3 eixos. Em X está no intervalo 10:60, ou seja é está limitado entre os pontos 10m e 60m, conferindo uma largura de 50m. Em Y está no intervalo 10:60, conferindo uma largura de 50m. Já em Z está no intervalo 0:10, conferindo uma altura de 10m.

Nas linhas 11 a 14 estão as especificações do objeto dinâmico, denominado esfera. Os valores indicam que o centro da esfera está a 10m da origem no eixo X, 10m  da origem no eixo Y e 50m da origem no eixo Z. O objeto possui raio=2.5m, massa=5kg e coeficiente de restituição = 0.25

Os atributos da esfera, colocam ela verticalmente alinhada com a caixa, em uma posição bem alta. A simulação consiste em deixar a esfera cair, atraída pela força da gravidade. A queda simulada é de 45m. <revisar esta informação>


\emph{explicar o AABB}

\subsubsection{Simulação de queda da esfera}

Com a execução da simulação descrita, a esfera é atraída em direção ao chão, devido à força da gravidade presente no ambiente, até que eventualmente colida com a caixa que está sobre o chão. Quando a colisão ocorre, a esfera recebe um impulso com sentido contrário à queda, que faz o objeto quicar algumas vezes. 

Este trecho da simulação pode ser verificado pela saída acessível, no Registro de Saída~\ref{lg:esferaM1}, que mostra apenas alguns pontos da simulação. O detalhamento dos eventos registrados encontra-se a seguir:

\begin{lstlisting}[frame=single,caption=Queda da esfera. \label{lg:esferaM1}]
	T: 0.20s, Dir: b, 	Pos: 10.00:10.00:49.61, Vel: 1.96m/s
	T: 2.60s, Dir: b, 	Pos: 10.00:10.00:14.29, Vel: 25.51m/s
	T: 2.80s, Dir: c, 	Pos: 10.00:10.00:12.50, Vel: 6.87m/s
	T: 3.00s, Dir: c, 	Pos: 10.00:10.00:13.87, Vel: 6.87m/s
\end{lstlisting}

<revisar esta lista> 
\begin{itemize}
	\item Linha 1: Iteração em 0.2s. A gravidade começa atrair a esfera para baixo;
	\item Linha 2: Iteração em 2.60s. A esfera desceu 35,32m em relação a posição da primeira iteração, estando agora com velocidade de 25.51m/s;
	\item Linha 3: Iteração em 2.80s. A esfera colidiu com a caixa, recebendo impulso de força contrária à queda, mudando de direção. Agora está subindo (na posição 12.50m) a uma velocidade de 6.87m/s; 
	\item Linha 4: Iteração no tempo 3.00s. A esfera ainda está subindo (agora na posição 13.87m) em função da colisão com a caixa. Este movimento permanece até que a gravidade consiga anular o impulso restituído, ou que outra força seja aplicada no objeto.
\end{itemize}

No instante 4.40s a esfera está parando de quicar, quando uma força lateral de $35m/s^{2}$ é aplicada, causando um deslocamento lateral para a direita (leste), até chegar ao limite caixa. Este momento da simulação pode ser verificado no Registro de Saída~\ref{lg:esfera_m2}. O detalhamento dos eventos registrados encontra-se a seguir:

\begin{lstlisting}[frame=single,caption=Esfera sobre a caixa. \label{lg:esfera_m2}]
	T: 4.40s, Dir: c, 	Pos: 10.00:10.00:12.50, Vel: 1.96m/s
	T: 4.60s, Dir: lc, 	Pos: 11.40:10.00:12.89, Vel: 7.27m/s
	T: 4.80s, Dir: l, 	Pos: 14.20:10.00:12.89, Vel: 14.00m/s
	T: 5.60s, Dir: l, 	Pos: 39.40:10.00:12.50, Vel: 42.00m/s
\end{lstlisting}
%% 	T: 5.80s, Dir: l, 	Pos: 49.20:10.00:12.50, Vel: 49.00m/s

\begin{itemize}
	\item Linha 1: Iteração em 4.40s. Instante que a força lateral é aplicada;
	\item Linha 2: Iteração em 4.60s. Esfera começa a mover-se também para leste. A esfera ainda está "quicando", mas em uma distância muito pequena da caixa (apenas 39cm). As duas forças causam um movimento na direção lc, para leste e acima, descrevendo uma pequena parábola;
	\item Linha 3: Iteração em 4.80s. Esfera estabilizou verticalmente sobre a caixa. Agora com movimento para leste, com velocidade 14m/s, devido a força aplicada desde o instante 4.40s; 
	\item Linha 4: Iteração em 5.60s. A esfera continua acelerando na direção leste em cima do cubo, agora na posição 39.40m com 42m/s.
\end{itemize}

A esfera continua sendo "empurrada" para leste, até o momento em que “sai de cima da caixa”, iniciando um movimento diagonal, para leste e abaixo. A força lateral ainda está agindo sobre o objeto. Este trecho da simulação pode ser verificado no Registro de Saída \ref{lg:esfera_m3}, que mostra apenas alguns instantes da simulação. O detalhamento é feito a seguir:

\begin{lstlisting}[frame=single,caption=Esfera se movendo diagonalmente\label{lg:esfera_m3}]
	T: 6.00s, Dir: l	Pos: 60.40:10.00:12.50, Vel: 56.00m/s
	T: 6.20S, Dir: lb 	Pos: 73.00:10.00:12.11, Vel: 63.03m/s
	T: 7.40S, Dir: c 	Pos: 178.00:10.00:2.50, Vel: 105.06m/s
	T: 8.00S, Dir: lb 	Pos: 249.40:10.00:3.38, Vel: 126.00m/s
	T: 8.60S, Dir: l, 	Pos: 333.40:10.00:2.50, Vel: 147.00M/s
\end{lstlisting}
%% 	T: 8.40S, Dir: c 	Pos: 304.00:10.00:2.50, Vel: 140.00m/s

\begin{itemize}
	\item Linha 1: Iteração me 6.00s. Esfera continua sua trajetória na direção leste, no entanto, seu centro ultrapassou a borda da caixa;
	
	\item Linha 2: Iteração em 6.20s. Esfera começa a ser puxada pela gravidade novamente. Agora, duas forças estão agindo neste objeto, fazendo com que descreva um movimento diagonal, na direção lb (leste e abaixo), com velocidade de 63.03m/s.
	
	\item Linha 3: Iteração em 7.40s. Esfera atingiu o chão pela primeira vez e continua a ser empurrada para leste.
	
	\item Linha 4: Iteração em 8.00s. Esfera está descrevendo um movimento de parábola e atinge sua altura máxima após quicar no chão. A direção está leste e baixo, com velocidade de 126m/s;
	
	\item Linha 5: Iteração em 8.60s. Esfera para de quicar, movimentando-se para leste, com velocidade crescente, até a borda do mundo.
\end{itemize}

\begin{lstlisting}[frame=single,caption=Esfera sobre o chão e saída do ambiente\label{lg:esferaM4}]
	T: 8.60S, Dir: l, Pos: 333.40:10.00:2.50, Vel: 147.00M/s
	T: 243.20S, Dir: l, Pos: 998791.00:10.00:2.50, Vel: 8358.00M/s
	T: 244.20S, Dir: lb, Pos: 1007170.00:10.00:-1.42, Vel: 8393.00M/s
	T: 500.01S, Dir: lb, Pos: 4300083.50:10.00:-323212.34, Vel: 17527.70M/s
\end{lstlisting}

Linha 1: Iteração no tempo 8.60s. A esfera se estabiliza no solo e apenas acelera na direção leste até o final do ambiente.
Linha 2: Iteração no tempo 243.20s. A esfera está em alta velocidade na direção leste e está próxima ao limite leste do ambiente.
Linha 3: Iteração no tempo 244.20s. A esfera ultrapassa os limites do ambiente e volta a ser puxada para baixo pela força da gravidade e continua em uma trajetória diagonal até o fim da simulação.
Linha 4: Iteração no tempo 500.01s. Última iteração da simulação, e a esfera se encontra em auta velocidade extremamente longe dos limites do ambiente de simulação.

\section{Considerações finais}
O Proto3d é uma aplicação desenvolvida com a finalidade de simular o movimento de um ou mais corpos sólidos, detectar e resolver as colisões entre estes corpos e ainda, apresentar a simulação em uma saída acessível a pessoas com deficiência visual.

No simulador, tanto o ambiente como os corpos sólidos, possuem parâmetros configuráveis de forma independente. Os parâmetros do ambiente são:  gravidade e limites. E os parâmetros dos corpos rígidos são: massa, velocidade, orientação, coeficiente de restituição, forma, forças, e outras propriedades Destaca-se que neste relato, para fins de simplificação, optou-se em restringir a forma dos objetos em esferas e caixas.

A simulação executada pelo Proto3d atingiu os objetivos estabelecidos. Os resultados de movimento mostrados no registro de saída podem ser validados com cálculos de física newtoniana. As colisões são detectadas e o movimento resultante também têm resultados coerentes com os esperados. 

Como trabalhos futuros no desenvolvimento do Proto3d, planeja-se implementar o rastreamento de múltiplos objetos, criar uma interface gráfica para tornar o simulador mais acessível, aprimorar o método de integração usado para calcular os movimentos dos objetos, e considerar outras propriedades físicas, como velocidade angular. Além disso, espera-se adicionar a opção configurável de introduzir arrasto do ar no ambiente de simulação.

\bibliographystyle{sbc}
\bibliography{sbc-template}

\end{document}
