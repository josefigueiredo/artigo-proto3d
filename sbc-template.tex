\documentclass[12pt]{article}

% ---
% PACOTES
% ---

% ---
% Pacotes fundamentais 
% ---
\usepackage{sbc-template}
\usepackage{graphicx,url}
\usepackage[utf8]{inputenc}
\usepackage[brazil]{babel}
% ---


\usepackage{caption}
\usepackage{listings}
\usepackage{xcolor}

\definecolor{codegreen}{rgb}{0,0.6,0}
\definecolor{codegray}{rgb}{0.5,0.5,0.5}
\definecolor{codepurple}{rgb}{0.58,0,0.82}
\definecolor{backcolour}{rgb}{0.95,0.95,0.92}

\lstdefinestyle{mystyle}{
    backgroundcolor=\color{backcolour},   
    commentstyle=\color{codegreen},
    keywordstyle=\color{magenta},
    numberstyle=\tiny\color{codegray},
    stringstyle=\color{codepurple},
    basicstyle=\ttfamily\footnotesize,
    breakatwhitespace=false,         
    breaklines=true,                 
    captionpos=t,                    
    keepspaces=true,                 
    numbers=left,                    
    numbersep=5pt,                  
    showspaces=false,                
    showstringspaces=false,
    showtabs=false,                  
    tabsize=2
}

\lstset { %s
    style=mystyle,
    language=C++,
  	%basicstyle=\footnotesize,% basic font setting
  	%basicstyle=\ttfamily
}


\renewcommand{\lstlistingname}{Código}
% --- 
% SEUS AJUSTES DEVEM SER FEITOS ABAIXO DESSE COMENTÁRIO
% --- 


% ---
% Pacotes de citações
% ---
%\usepackage[brazilian,hyperpageref]{backref}	 % Paginas com as citações na bibl
%\usepackage[alf]{abntex2cite}	% Citações padrão ABNT


\sloppy

\title{Aqui vai o título}

\author{Roger \inst{1}, José Antônio de Figueiredo\inst{1}}

\address{Curso Bacharel em Ciência da Computação -- Instituto Federal Sul-rio-grandense (IFSUL)\\
	Passo Fundo -- RS -- Brasil
	\email{rogerlash@gmail.com, josefigueiredo@ifsul.edu.br https://orcid.org/0009-0005-5101-516X}
}

\begin{document} 

\maketitle

\begin{abstract}
\end{abstract}
     
\begin{resumo} 
  Aqui vai o resumo
\end{resumo}

\section{Introdução}

\section{Simulação Computacional}

Segundo \cite{duran2018computer} a simulação computacional é uma técnica amplamente utilizada em diversas áreas, como por exemplo no estudo de sistemas complexos que por diversos fatores, não podem ser facilmente reproduzidos.

A simulação permite avaliar diversos cenários de uma pesquisa, apenas alterando suas variáveis de controle, reduzindo assim os custos e os riscos de um experimento físico. Conforme \cite{marschner2018}, essa técnica pode ser aplicada em diversas áreas como biologia, engenharia, física, química, entre outras.

Para construir uma simulação, é necessário uma representação matemática e/ou computacional do comportamento do sistema a ser simulado.

\subsection{Formas Geométricas e Movimento}
A técnica usada para representar uma forma geométrica computacionalmente dependerá do nível de detalhamento desejado \cite{pharr2016}. É possível fazer esta representação de diversas maneiras, por exemplo: forma vetorial, paramétrica ou de forma poligonal. Neste trabalho, adota-se a representação vetorial, com coordenadas cartesianas tridimensionais.

Para fins de simplificação, este trabalho limita-se a modelar objetos na forma de esferas e caixas. Esferas são modeladas com um ponto (centro da esfera) e um raio, enquanto caixas são modeladas com um ponto mínimo e um ponto máximo, que representam o canto inferior esquerdo, e o canto superior direito da caixa.

No contexto desta simulação, os objetos podem ser categorizados em duas classes principais: Objetos estáticos e objetos dinâmicos. Os objetos estáticos não se movem por conta própria e nem sob influência de forças externas. Em contrapartida, os objetos dinâmicos podem se mover por conta própria, bem como em resposta a forças externas.

\subsection{Colisão e Detecção de Colisão}

Uma colisão ocorre quando dois ou mais objetos entram em contato, compartilhando algum espaço do ambiente simulado~\cite{ericson2004real}. Segundo~\cite{bourg2013physics}, a colisão envolve cálculos para determinar as consequências/efeitos desse contato.

Segundo~\cite{bourg2013physics} a detecção de colisão é um problema de geometria computacional, que procura determinar \emph{se} e \emph{onde}, dois ou mais objetos colidiram. Pode ser detectada de diversas maneiras, dependendo da natureza dos objetos envolvidos~\cite{foley1996}, por exemplo: 1) se a distância entre os dois objetos está abaixo de uma determinada tolerância; 2) se os objetos estiverem sobrepondo-se em um ou mais pontos.

O processo para detecção de colisão é dividido em fase de previsão e de identificação. A previsão busca objetos que possivelmente estejam colidindo, enquanto a identificação localiza as colisões ocorridas. A separação em duas fases distintas garante um consumo racional do recurso computacional. 

Após detectar uma colisão entre dois objetos, torna-se necessário resolver o evento, atualizando a posição da simulação. Em simulações que não requerem realismo, apenas a correção das posições dos objetos é suficiente. Já em simulações mais avançadas, outros métodos podem ser necessários, como a resolução de colisão por impulso discutida em~\cite{bourg2013physics} e~\cite{baraff1992dynamic}, onde as características físicas dos objetos são considerados para que a reação seja a mais adequada possível.

%########
\section{Materiais e Métodos}
-> objetivo do simulador
-> interface de configuração (falar somente do que pode ser configurado.)
-> principais classes (sem detalhar todos os atributos)
-> recursos utilizados

%########

O Proto3d é um programa que permite realizar simulações de corpo rígido.
Ele possui algumas propriedades configuráveis tais como gravidade, e limites do ambiente.
No ambiente, objetos são adicionados para se realizar a simulação. Estes objetos são normalmente chamados de corpos rígidos.
Os corpos rígidos, são compostos com diversas propriedades como restituição, posição, velocidade, orientação, forma, e outras propriedades dependendo do nível de detalhamento que se deseja.
O simulador, é constituído  por diversos componentes:

\begin{itemize}
\item Mundo: É o ambiente virtual onde a simulação ocorre. É o coração de todo o mecanismo.
\item Corpo rígido: representa um objeto no ambiente, seja ele dinãmico ou estático.
\item Pré-filtragem: No início do quadro, filtra uma lista de objetos preparando-os para a movimentação.
\item Integração numérica: Resolve as equações do movimento para cada objeto preparado pela pré-filtragem.
\item Fase ampla: verifica uma lista de pares de objetos cujo seus AABBS se sobreponham, o que pode indicar que o par de objetos pode ou não estar colidindo.
\item Fase estreita: aplica testes de intercessão entre geometrias específicas para confirmar ou descartar a colisão, que a fase ampla suspeitou.
\item Resolvedor de colisões: Atua separando  os objetos caso aja sobreposição. Esta etapa trabalha no conjunto de objetos retornado pela fase estreita.
\item Despache de eventos: Nesta faze, notificações são enviadas para o usuário para informar sobre o que aconteceu no quadro atual.
\end{itemize}

\section{Resultados Obtidos}
O principal resultado é a aplicação Proto3d, que executa simulações de movimento e colisão de corpos rígidos em duas ou três dimensões, dentro do ambiente configurado pelo usuário. A simulação do movimento é implementada na dinâmica newtoniana, enquanto a detecção de colisão é implementada em uma estrutura de dados \emph{Octree}, com algoritmos \emph{Broad Phase} \emph{Narrow Phase} para previsão e identificação, respectivamente.

Além disso, o Proto3d é uma ferramenta acessível às pessoas com deficiência visual, desta forma os resultados das simulações são apresentados no Registro de Saída, um log acessível aos leitores de tela disponíveis no mercado.

\subsection {Proto3d}
O Proto3d, é uma aplicação para simulação de movimento e colisão, ainda em fase de protótipo. A ferramenta permite adicionar/remover e configurar corpos rígidos, bem como controlar os seguintes parâmetros: Gravidade; Limites Físicos do ambiente; Passo de tempo. 

A execução da simulação é apresentada em formato texto no Registro de Saída, sendo que esta versão do Proto3d não dispõe de interface gráfica. Para validação das simulações, foram executados alguns testes com grandezas conhecidas. 

Os objetos simulados no Proto3d são chamados de corpos rígidos, que podem ser estáticos ou dinâmicos. Corpos dinâmicos são objetos que possuem forma, massa, posição, velocidade e restituição. Corpos estáticos são um objetos que possuem forma, posição e orientação, desprezando propriedades como massa, restituição e velocidade. 

Para este estudo, optou-se em restringir o formato dos objetos em esferas e caixas, apresentando os resultados da simulação para um destes testes.


\subsection{Registro de Saída}
O Proto3d apresenta os resultados de cada simulação no Registro de Saída, no formato de log de texto e também armazena em um arquivo para análise posterior. O registro é organizado em duas partes principais: \emph{Visão geral do ambiente} e \emph{Rastreio de movimento do objeto}.

\noindent\textbf{Visão geral do ambiente}:  

Na parte inicial do Registro de Saída, são apresentadas as configurações gerais do ambiente de simulação como gravidade, quantidade de objetos simulados, bem como uma descrição detalhada de cada objeto no ambiente. 

O Registro de Saída~\ref{lg:init_world} mostra inicialização do ambiente para o teste proposto. Na linha 1 informa gravidade ativada, com valor $ 9,81 m/s^{2} $ (linha 2); Nas linhas 4,6 apresentam as configurações do chão. Para fins de adequação ao formato do artigo, foram omitidas algumas linhas do registro original. 

\begin{lstlisting}[frame=single,caption=Inicialização do ambiente\label{lg:init_world}]
	Tempo atual: 0.00s; Último passo de tempo: 0.00; Gravidade: True
	Gravidade: 0.00:0.00:-9.81
	Limites: 0.00:1000000.00,0.00:1000000.00,-10000.00:1000000.00
	
	Nome: Chão; Pos: 500000.00:500000.00:-5000.00; Forma Geom: Caixa
	AABB -> X: 0.00:1000000.00; Y: 0.00:1000000.00; Z: -10000.00:0.00 
	
	Nome: Caixa estática; Pos: 35.00:35.00:5.00; Forma Geom: Caixa
	AABB -> X: 10.00:60.00; Y: 10.00:60.00; Z: 0.00:10.00; 
	
	Nome: Esfera dinâmica; Forma Geo: esfera; 
	Centro: 10.00:10.00:50.00; Raio: 2.50
	Masss: 5.00; Restituição: 0.25
	AABB -> X: 7.50:12.50; Y: 7.50:12.50; Z: 47.50:52.50
\end{lstlisting}

Por convenção, as grandezas dos objetos podem ser lidas como: dimensão em metros (m), massa em quilogramas (kg), tempo em segundos (s).

\noindent\textbf{Rastreio de movimento:}

Após o início da simulação, o Proto3d registra as informações para rastreio de movimento do objeto dinâmico em análise. Os registros são feitos a cada passo de tempo, apresentando as seguintes informações:

\begin{itemize}
\item T: Informa a quantidade de tempo decorrido desde o início da simulação até este registro do rastreio.
\item Dir: Informa a direção (ou trajetória) do objeto simulado conforme a seguinte legenda: s (direção sul), n (direção norte), o (direção oeste), l (direção leste), c (direção acima/para cima), b (direção abaixo/para baixo).

\item Pos: Informa o lugar onde o objeto está no instante do registro. Todos os objetos são localizados nos eixos X (horizontal), Y (profundidade) e Z (vertical).

\item Vel: Informa a velocidade do objeto no instante do registro representada em m/s.
\end{itemize}

\subsection{Simulação de queda da esfera }

Esta simulação demonstra a queda de uma esfera sobre uma caixa que está sobre o chão do ambiente, "rolando à direita" e caindo novamente até atingir o chão. 

A esfera é um objeto dinâmico e a caixa um objeto estático. O Registro de Saída \ref{lg:init_world} mostra, nas linhas 11 à 14, as especificações da esfera: Posição inicial x=10, z=10 e y=50; raio 2.5; massa 5 e restituição 0.25. Nas linhas 8 e 9 mostra as especificações da caixa: Posição inicial x entre 10, e 60, y entre 10, 60, e z entre 0, 10;

O teste consiste em posicionar uma esfera, verticalmente alinhada com a caixa estática, mas em uma posição bem alta e deixar esta esfera cair. A altura da queda é de 40m até atingir o topo da caixa.

Com o avanço da simulação, a esfera é atraída em direção ao chão, devido à força da gravidade presente no ambiente, até que eventualmente colida com a caixa  que está sobre o chão. Quando a colisão ocorre, a esfera recebe um impulso com sentido contrário à queda, que faz o objeto quicar algumas vezes. 

Este trecho da simulação pode ser verificado na saída acessível, no Registro de Saída~\ref{lg:esferaM1}, que mostra apenas alguns pontos relevantes do rastreio.

\begin{lstlisting}[frame=single,caption=Queda da esfera. \label{lg:esferaM1}]
	T: 0.20s, Dir: b, Pos: 10.00:10.00:49.61, Vel: 1.96m/s
	T: 2.60s, Dir: b, Pos: 10.00:10.00:14.29, Vel: 25.51m/s
	T: 2.80s, Dir: c, Pos: 10.00:10.00:12.50, Vel: 6.87m/s
	T: 3.00s, Dir: c, Pos: 10.00:10.00:13.87, Vel: 6.87m/s
\end{lstlisting}

O detalhamento dos eventos registrados é feito a seguir:
\begin{itemize}
	\item Linha 1: Iteração em 0.2. A força da gravidade começa atrair a esfera para baixo;
	\item Linha 2: Iteração em 2.60s. A esfera caiu 35,32m em relação à posição da primeira iteração, estando agora com velocidade de 25.51m/s;
	\item Linha 3: Iteração em 2.80s. A esfera colidiu com a caixa estática, recebendo uma força contrária a queda, mudando de direção e agora está subindo (na posição 12.50m) a uma velocidade de 6.87m/s; 
	\item Linha 4: Iteração no tempo 3.00s. A esfera ainda está subindo (agora na posição 13.87m) em função da colisão com a caixa. Este movimento permanece até que a gravidade consiga anular o impulso aplicado e comece a atrair a esfera novamente para baixo.
\end{itemize}

No instante 4.40s uma nova força com $35m/s^{2}$ é aplicada na esfera, deslocando-a para direita até que saia de cima da caixa. Este momento da simulação pode ser verificado no Registro de Saída~\ref{lg:esfera_m2}. 

\begin{lstlisting}[frame=single,caption=Esfera sobre a caixa. \label{lg:esfera_m2}]
	T: 4.40s, Dir: c, 	Pos: 10.00:10.00:12.50, Vel: 1.96m/s
	T: 4.60s, Dir: lc, 	Pos: 11.40:10.00:12.89, Vel: 7.27m/s
	T: 4.80s, Dir: l, 	Pos: 14.20:10.00:12.89, Vel: 14.00m/s
	T: 5.60s, Dir: l, Pos: 39.40:10.00:12.50, Vel: 42.00m/s
	T: 5.80s, Dir: l, Pos: 49.20:10.00:12.50, Vel: 49.00m/s
\end{lstlisting}

O detalhamento dos eventos registrados é feito a seguir:

\begin{itemize}
	\item Linha 1: Iteração em 4.40s. Instante que a força lateral é aplicada;
	\item Linha 2: Iteração em 4.60s. Esfera começa a mover também para a direita. Observar que a esfera ainda está "quicando" mas numa altura muito pequena da caixa (apenas 39cm). Isto provoca um movimento na direção lc, para leste e acima, descrevendo uma parábola;
	\item Linha 3: Iteração em 4.80s. A esfera estabilizou verticalmente sobre a caixa. Apenas a força horizontal (para direita) está agindo, deslocando a mesma para o lado (eixo X está aumentando) e a velocidade está em 14m/s;
	\item Linha 4: Iteração em 5.60s. A esfera continua acelerando  na direção leste em cima da caixa.
	\item Linha 5: Iteração em 6.00s. A esfera continua sua trajetória na direção leste, no entanto, seu centro ultrapassou a borda da caixa menor.
\end{itemize}

 - linha 1. Pode-se observar que neste instante a esfera está movendo em duas direções: leste e cima, a força lateral foi aplicada 

As linhas 

A força aplicada leva a esfera até os limites da caixa estática. 




No momento em que a esfera “sai de cima da caixa”, a gravidade atrai novamente a esfera em direção ao chão. Deste ponto em diante a esfera está sujeita a duas forças, uma vertical (gravidade) e outra horizontal, uma trajetória descendente na diagonal até chegar ao chão, quando o momento vertical cessa novamente.



Ao tocar o chão, a esfera quica novamente, mas agora está sob influência de uma força lateral, descreve um movimento diagonal. No momento em que a esfera parar de quicar, será mantido o movimento lateral até que esta saia do ambiente da simulação.
 Tempo: 4.40S, Direção: c
Posição atual: 10.00:10.00:12.50, Velocidade: 1.96M/S


Linha 1: Iteração no tempo 4.80s. A autura do salto da esfera se tornou pequeno o suficiente para ativar a força horizontal que empurra a esfera para a direita.

Linha 2: Iteração no tempo 5.60s. A esfera se estabilizou verticalmente em cima do cubo e apenas a força horizontal está agindo sobre ela.

Linha 3: Iteração no tempo 5.80s. A esfera continua acelerando  na direção leste em cima do cubo.

Linha 4: Iteração no tempo 6.00s. A esfera continua sua trajetória na direção leste, no entanto, seu centro ultrapassou a borda do cubo menor.

\begin{lstlisting}[frame=single,caption=Esfera se movendo diagonalmente\label{lg:esfera_m3}]
T: 6.20S, Dir: lb Pos: 73.00:10.00:12.11, Vel: 63.03M/s
T: 7.40S, Dir: c Pos: 178.00:10.00:2.50, Vel: 105.06M/s
T: 8.00S, Dir: lb Pos: 249.40:10.00:3.38, Vel: 126.00M/s
T: 8.40S, Dir: c Pos: 304.00:10.00:2.50, Vel: 140.00M/s
\end{lstlisting}

Linha 1: Iteração no tempo 6.20s. A esfera começa a ser puxada para baixo novamente já que não está mais apoiada sobre o cubo menor. Agora tem duas forças agindo sobre a esfera.

Linha 2: Iteração no tempo 7.40s. A esfera atingiu o chão pela primeira vez e continua a ser empurrada para leste.

Linha 3: Iteração no tempo 8.00s. A esfera atinge sua autura máxima após atingir o solo e quicar.

Linha 4: Iteração no tempo 8.40s. A esfera atinge o solo pela segunda vez, e continua na sua trajetória leste.

\begin{lstlisting}[frame=single,caption=Esfera sobre o chão e saída do ambiente\label{lg:esferaM4}]
T: 8.60S, Dir: l, Pos: 333.40:10.00:2.50, Vel: 147.00M/s
T: 243.20S, Dir: l, Pos: 998791.00:10.00:2.50, Vel: 8358.00M/s
T: 244.20S, Dir: lb, Pos: 1007170.00:10.00:-1.42, Vel: 8393.00M/s
T: 500.01S, Dir: lb, Pos: 4300083.50:10.00:-323212.34, Vel: 17527.70M/s
\end{lstlisting}

Linha 1: Iteração no tempo 8.60s. A esfera se estabiliza no solo e apenas acelera na direção leste até o final do ambiente.
Linha 2: Iteração no tempo 243.20s. A esfera está em auta velocidade na direção leste e está próxima ao limite leste do ambiente.
Linha 3: Iteração no tempo 244.20s. A esfera ultrapassa os limites do ambiente e volta a ser puxada para baixo pela força da gravidade e continua em uma trajetória diagonal até o fim da simulação.
Linha 4: Iteração no tempo 500.01s. Última iteração da simulação, e a esfera se encontra em auta velocidade extremamente longe dos limites do ambiente de simulação.

\section{ Considerações finais:}
O Proto3d é uma aplicação desenvolvida com a finalidade de simular o movimento de um ou mais corpos sólidos, detectar e resolver as colisões entre estes corpos e ainda, apresentar a simulação em uma saída acessível a pessoas com deficiência visual.

No simulador, tanto o ambiente como os corpos sólidos, possuem parâmetros configuráveis de forma independente. Os parâmetros do ambiente são:  gravidade e limites. E os parâmetros dos corpos rígidos são: massa, velocidade, orientação, coeficiente de restituição, forma, forças, e outras propriedades Destaca-se que neste relato, para fins de simplificação, optou-se em restringir a forma dos objetos em esferas e caixas.

A simulação executada pelo Proto3d atingiu os objetivos estabelecidos. Os resultados de movimento mostrados no registro de saída podem ser validados com cálculos de física newtoniana. As colisões são detectadas e o movimento resultante também têm resultados coerentes com os esperados. 

Como trabalhos futuros podem ser listados: 1) Simulação gráfica, 2)Estrutura para rastrear múltiplos objetos, etc…



\section{References}

Bibliographic references must be unambiguous and uniform.  We recommend giving
the author names references in brackets, e.g. \cite{knuth:84},
\cite{boulic:91}, and \cite{smith:99}.

The references must be listed using 12 point font size, with 6 points of space
before each reference. The first line of each reference should not be
indented, while the subsequent should be indented by 0.5 cm.

\bibliographystyle{sbc}
\bibliography{sbc-template}

\end{document}
